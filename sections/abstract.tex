Recovering Construction Trees from potentially noisy point clouds is an important aspect of Reverse Engineering tasks in Computer Aided Design. 
Solutions based on algorithmic geometry impose constraints on usable model representations and noise robustness. 
Re-formulating the problem as a combinatorial optimization problem and solving it with an Evolutionary Algorithm mitigates these constraints at the cost of increased computation times. 
This paper proposes a detailed analysis of the associated optimization problem and a search space partitioning scheme that is able to accelerate Evolutionary Algorithm based Construction Tree recovery while exploiting parallelization capabilities of modern CPUs.
The evaluation indicates a speed-up of up to $14.3x$ compared to the baseline approach while resulting tree sizes increase by TODO$\%$ on average.    
